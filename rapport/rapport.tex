% Preamble
\documentclass[12pt]{amsart}
\usepackage{ottostyle}

\sloppypar

% Bibliography
\addbibresource{references.bib}
\begin{document}


\section{Argument}
We will now describe how to construct \( \operatorname{soc}D \Gamma \) given \( \underline{r} \Gamma \), and the other way around.

Suppose that we know \( \underline{r}\Gamma \).
Then we also know \( \operatorname{Top} \Gamma \), and we have a canonical projection
\[
\xymatrix{
	\Gamma \ar[r]^p & \operatorname{Top} \Gamma .
}
\]
Applying \( D \) we get a map
\[
\xymatrix{
	D \operatorname{Top} \Gamma \ar[r]^{p^*} & D \Gamma ,
}
\]
and \( \im p^* = \operatorname{Soc} D \Gamma \) \textcolor{red}{reference} .

Conversely, suppose that we know \( \operatorname{Soc} D \Gamma \).
This is a simple right \( \Gamma \)-module \textcolor{red}{reference} .
So if \( f \in \operatorname{Soc} D \Gamma \) is any non-zero element, we obtain a short exact sequence of right \( \Gamma \)-modules
\[
\xymatrix{
	0 \ar[r] & M \ar[r] & \Gamma \ar[r]^{\phi} & \operatorname{Soc} D \Gamma \ar[r] & 0 ,
}
\]
where \( \phi (a) = f \cdot a \) and \( M = \ker \phi \).
Since \( \operatorname{Soc} D \Gamma \) is simple, \( M \) is a maximal right ideal.
But \( \Gamma \) has only one right maximal ideal \textcolor{red}{reference} , so \( M = \underline{r} \Gamma \).


\end{document}
